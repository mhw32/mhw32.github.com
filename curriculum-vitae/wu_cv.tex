%----------------------------------------------------------------------------------------
%	PACKAGES AND OTHER DOCUMENT CONFIGURATIONS
%----------------------------------------------------------------------------------------

\documentclass[margin, 10pt]{res} % Use the res.cls style, the font size can be changed to 11pt or 12pt here

\usepackage{helvet} % Default font is the helvetica postscript font
%\usepackage{newcent} % To change the default font to the new century schoolbook postscript font uncomment this line and comment the one above
\usepackage{hyperref}

\setlength{\textwidth}{5.1in} % Text width of the document

\begin{document}

%----------------------------------------------------------------------------------------
%	NAME AND ADDRESS SECTION
%----------------------------------------------------------------------------------------

\moveleft.5\hoffset\centerline{\large\bf Mike Wu} % Your name at the top

\moveleft\hoffset\vbox{\hrule width\resumewidth height 1pt}\smallskip % Horizontal line after name; adjust line thickness by changing the '1pt'
% \moveleft.5\hoffset\centerline{334 Jordan Hall}
\moveleft.5\hoffset\centerline{(858) 740-9967}
\moveleft.5\hoffset\centerline{wumike@stanford.edu}
\moveleft.5\hoffset\centerline{\url{https://www.mikehwu.com}}

%----------------------------------------------------------------------------------------

\begin{resume}

%----------------------------------------------------------------------------------------
%	Education SECTION
%----------------------------------------------------------------------------------------

\section{Education}

\textbf{Stanford University} \hfill 2017 - Present \\
Ph.D., M.S. in Computer Science \\
Advisor: Noah Goodman \\
Select Courses: CS142 (Web Applications), CS228 (Probabilistic Graphical Models), CS428 (Computation and Cognition), CS242 (Programming Languages), CS251 (Cryptocurrencies and Blockchain Technologies), Math120 (Group Theory), Math115 (Functions of a Real Variable), Math171 (Fundamental Concepts of Analysis), Math175 (Elementary Functional Analysis), Math205a (Measure Theory)

\textbf{Yale University}, Hopper College \hfill 2012 - 2016 \\
B.S. with Distinction in Computer Science \\
Yale college council, science committee \\
Select Courses: CPSC202 (Discrete Math), CPSC223 (Data Structures and Algorithms), CPSC323 (Systems Programming and Computer Arch.), CPSC437 (Databases), CPSC467 (Cryptography and Computer Security)

\textbf{University of Oxford}, New College \hfill 2015 \\
1st mark in three courses in Computer Science\\
Oxford computing society \\
Select Courses: Advanced Data Structures and Algorithms, Machine Learning

%----------------------------------------------------------------------------------------
%	Awards SECTION
%----------------------------------------------------------------------------------------

\section{Awards and \\ Honors}

Botha-Chan Innovation Fellow \hfill 2021 \\
Pear VC Fellow \hfill 2020 \\
Educational Data Mining Best Paper Award \hfill 2020 \\
Educational Data Mining Best Student Paper Runner-up \hfill 2020 \\
AAAI Outstanding Student Paper Award \hfill 2019 \\
IDEO CoLab Fellow \hfill 2019 \\
AngelHack, Augmented Reality Category 1st place \hfill 2018 \\
AMIA CRI Nominated for Informatics Award \hfill 2017 \\
API World Hackathon, Telesign API 1st place \hfill 2017 \\
Trueface.ai Hackathon, 1st place \hfill 2017 \\
HackMIT Top 8 Hacks, Dropbox API 1st place \hfill 2015 \\
Qualcomm QLiving Scholarship \hfill 2014 \\
Intel ISEF Semifinalist \hfill 2012 \\
Siemens Competition Semifinalist \hfill 2012 \\
Intel ISEF Finalist, 3rd place in Computer Science \hfill 2011 \\
XSEDE Best Student Poster \hfill 2011 \\

\section{Fellowships}
Stanford Interdisciplinary Graduate Fellowship (Karr Family Fellow) \hfill 2020-Current \\
NSF Graduate Research Fellowship \hfill 2017-2020 \\

%----------------------------------------------------------------------------------------
%   In submission SECTION
%----------------------------------------------------------------------------------------

% \section{Submitted / \\ Under Review}

%----------------------------------------------------------------------------------------
%   Published SECTION
%----------------------------------------------------------------------------------------

\section{Preprints}

\textbf{Mike Wu}, Richard L. Davis, Benjamin W. Domingue, Chris Piech, Noah Goodman. Modeling Item Response Theory with Stochastic Variational Inference. In \textit{ArXiv}, 2021.

\textbf{Mike Wu}, Chris Piech, Noah Goodman, Chelsea Finn. ProtoTransformer: A Meta-Learning Approach to Providing Student Feedback. In \textit{ArXiv}, 2021.

\section{Conference \\and Journal \\Proceedings}

\textbf{Mike Wu}, Noah Goodman, Stefano Ermon. Improving Compositionality of Neural Networks by Decoding Representations to Inputs. In \textit{Proc. 35th Annual Conference on Neural Information Processing Systems} (NeurIPS), 2021.

Ali Malik, \textbf{Mike Wu}, Vrinda Vasavada, Jinpeng Song, John Mitchell, Noah Goodman, Chris Piech. Generative Grading: Neural Approximate Parsing for Automated Student Feedback. In \textit{Educational Data Mining} (EDM), 2021. 

\textbf{Mike Wu}, Sonali Parbhoo, Michael C. Hughes, Ryan Kindle, Leo Celi, Maurizio Zazzi, Volker Roth, Finale Doshi-Velez. Regional Tree Regularization for Interpretability in Black Box Models. In \textit{Journal of Artificial Intelligence Research} (JAIR), 2021.

Alex Tamkin, \textbf{Mike Wu}, and Noah Goodman. Viewmaker Networks: Learning Views for Unsupervised Representation Learning. In \textit{Proc. 9th International Conference on Learning Representations} (ICLR), 2021.

\textbf{Mike Wu}, Chengxu Zhuang, Milan Mosse, Daniel Yamins, and Noah Goodman. Conditional Negative Sampling for Contrastive Learning of Visual Representations. In \textit{Proc. 9th International Conference on Learning Representations} (ICLR), 2021.

\textbf{Mike Wu} and Noah Goodman. A Simple Framework for Uncertainty in Contrastive Learning. In \textit{ArXiv}, 2021. 

\textbf{Mike Wu}, Chengxu Zhuang, Milan Mosse, Daniel Yamins, Noah Goodman. On Mutual Information in Contrastive Learning for Visual Representations. In \textit{ArXiv}, 2020.

\textbf{Mike Wu}, Richard L. Davis, Benjamin W. Domingue, Chris Piech, Noah Goodman. Variational Item Response Theory: Fast, Accurate, and Expressive. In \textit{Educational Data Mining} (EDM), 2020. [\textbf{Oral Presentation (15 min).}] [\textbf{Best Paper Award.}]

\textbf{Mike Wu}, Sonali Parbhoo, Michael C. Hughes, Volker Roth, Finale Doshi-Velez. Optimizing for Interpretability in Deep Neural Networks with Simulable Decision Trees. In \textit{Proc. 34rd AAAI Conference on Artificial Intelligence} (AAAI), 2020. [\textbf{Oral Presentation (20 min).}]

\textbf{Mike Wu}, Kristy Choi, Noah Goodman, Stefano Ermon. Meta-Amortized Variational Inference and Learning. In \textit{Proc. 34rd AAAI Conference on Artificial Intelligence} (AAAI), 2020.

\textbf{Mike Wu}, Noah Goodman. Multimodal Generative Models for Compositional Representation Learning. In \textit{ArXiv}, 2019.

Judith Fan, Robert X.D. Hawkins, \textbf{Mike Wu}, Noah Goodman. Pragmatic inference and Visual Abstraction Enable Contextual Flexibility during Visual Communication. In \textit{Computational Brain and Behavior} (COBB), 2019.

\textbf{Mike Wu}, Noah Goodman, Stefano Ermon. Differentiable Antithetic Sampling for Variance Reduction in Stochastic Variational Inference. In \textit{Proc. 22nd International Conference on Artificial Intelligence and Statistics} (AISTATS), 2019.

\textbf{Mike Wu}, Milan Mosse, Noah Goodman, Chris Piech. Zero Shot Learning for Code Education: Rubric Sampling with Deep Learning Inference. In \textit{Proc. 33rd AAAI Conference on Artificial Intelligence} (AAAI), 2019. [\textbf{Oral Presentation (12 min).}] [\textbf{Outstanding Student Paper Award.}]

\textbf{Mike Wu}, Noah Goodman. Multimodal Generative Models for Scalable Weakly-Supervised Learning. In \textit{Proc. 32nd Annual Conference on Neural Information Processing Systems} (NeurIPS), 2018.

\textbf{Mike Wu}, Michael C. Hughes, Sonali Parbhoo, Maurizio Zazzi, Volker Roth, Finale Doshi-Velez. Beyond Sparsity: Tree Regularization of Deep Models for Interpretability. In \textit{Proc. 32nd AAAI Conference on Artificial Intelligence} (AAAI), 2018. [\textbf{Spotlight Presentation (2 min).}]

Marzyeh Ghassemi, \textbf{Mike Wu}, Michael C. Hughes, Finale Doshi-Velez. Predicting Intervation Onset in the ICU with Switching Statespace Models. In \textit{Proc. AMIA Summit on Clinical Research Informatics} (CRI), 2017. [\textbf{Nominated for Informatics Award.}]

\textbf{Mike Wu}, Marzyeh Ghassemi, Mengling Feng, Leo Anthony Celi, Peter Szolovitz, Finale Doshi-Velez. Understanding Vassopressor Intervention and Weaning: Risk Prediction in a Public Heterogeneous Clinical Time Series Database. In \textit{Journal of the American Medical Informations Association, Volume 24, Issue 3, No. 1} (JAMIA), 2016.

\textbf{Mike Wu}, Madhu Krishnan. Edge-based Crowd Detection from Single Image Datasets. In \textit{International Journal of Computer Science Issues, Volume 12, Issue 1, No. 1} (IJCSI), 2013.

Madhu Krishnan, \textbf{Mike Wu}, Young Kang, Sarah H. Lee. Autonomous Mapping and Navigation through Utilization of Edge-based Optical Flow and Time-to-Collision. In \textit{ARPN Journal of Engineering and Applied Sciences, Volume 7, No. 12}, 2012.

\section{Workshops}

Oliver Zhang, \textbf{Mike Wu}, Jasmine Bayrooti, Noah Goodman. Temperature as Uncertainty in Contrastive Learning. In \textit{NeurIPS 2021 Workshop on Self-Supervised Learning Theory and Practice}.

Ananya Karthik, \textbf{Mike Wu}, Noah Goodman, Alex Tamkin. Tradeoffs Between Contrastive and Supervised Learning: An Empirical Study. In \textit{NeurIPS 2021 Workshop on Self-Supervised Learning Theory and Practice}.

\textbf{Mike Wu}, Jonathan Nafziger, Anthony Scodary, Andrew Maas. HarperValleyBank: A Domain-Specific Spoken Dialog Corpus. In \textit{Workshop on Machine Learning in Speech and Language Processing} (MLSLP), 2021.

\textbf{Mike Wu}, Kristy Choi, Noah Goodman, Stefano Ermon. Meta-Amortized Variational Inference and Learning. \textit{NeurIPS 2019 Workshop on Bayesian Deep Learning}. [\textbf{Spotlight Presentation (1 min).}]

\textbf{Mike Wu}, Sonali Parbhoo, Finale Doshi-Velez. Beyond Sparsity: Tree Regularization of Deep Models for Interpretability. \textit{NeurIPS 2017 Workshop on Transparent and interpretable Machine Learning in Safety Critical Environments}. [\textbf{Contributed Talk (10 min).}]

%----------------------------------------------------------------------------------------
%   PATENTS SECTION
%----------------------------------------------------------------------------------------

\section{Patents}

Frank Wood, \textbf{Mike Wu}, Yura Perov, Hongseok Yang. Computing engine, software, system and method. US Patent App. 15/465,131, 2017.

%----------------------------------------------------------------------------------------
%   PRESS SECTION
%----------------------------------------------------------------------------------------

\section{Press}

\textit{Can A.I. Grade Your Next Test?} The New York Times (07/20/2021).

\textit{First-of-its-kind Stanford machine learning tool streamlines student feedback process for computer science professors.} Stanford News (07/27/2021).

%----------------------------------------------------------------------------------------
%   TEACHING SECTION
%----------------------------------------------------------------------------------------

\section{Teaching \\ Experience}

\textbf{Teaching Assistant}, Dept. of Computer Science, Stanford University \hfill Winter 2021\\
CS109: Probability for Computer Scientists (Chris Piech)

\textbf{Teaching Assistant}, Dept. of Engineering, Stanford University \hfill Fall 2021\\
ME208: Patent Law and Strategy for Innovators and Entrepreneurs (Jeffrey Schox)

\textbf{Teaching Assistant}, Dept. of Computer Science, Stanford University \hfill Spring 2020\\
CS224S: Spoken Language Processing (Andrew Maas)\\
Award for Outstanding Work by Course Assistants (Top 5\%)

\textbf{Teaching Assistant}, Dept. of Computer Science, Stanford University \hfill Fall 2019\\
CS398: Computational Education (Chris Piech)

\textbf{Teaching Assistant}, Dept. of Computer Science, Yale University \hfill Spring 2016\\
CPSC437: Operating System Concepts (Avi Silberschatz)

\textbf{Teaching Assistant}, School of Management, Yale University \hfill Fall 2015\\
MGT656: Management of Software Development (Kyle Jensen)

%----------------------------------------------------------------------------------------
%	TALKS SECTION
%----------------------------------------------------------------------------------------

\section{Invited Talks}

Multidisciplinary University Research Initiative (MURI) Visual Common-Sense Annual Meeting, 2020.

Berkeley Center for Human-Compatible AI Technical Seminar, 2020.

Stanford Computer Forum Annual Meeting, 2019.

Human-Centered Artificial Intelligence Institute Symposium, 2019.

Yale Technology Conference, Yale University, 2016.

Probabilistic Programming Workshop, University of Southampton, 2016.

\section{Conference \\ Abstracts}

Judith Fan, Robert X.D. Hawkins, \textbf{Mike Wu}, Noah Goodman. Modeling contextual flexibility in visual communication. \textit{Vision Sciences Society Annual Meeting} (VSS), 2018.

William Smith, \textbf{Mike Wu}, Yura Perov, Frank Wood, Hongseok Yang. Spreadsheet probabilistic programming. \textit{Inaugural International Conference on Probabilistic Programming.} (PROBPROG), 2018.

Jessi Cisewski-Kehe, \textbf{Mike Wu}, Brittany Fasy, Wojciech Hellwing, Mark Lovell, Alessandro Rinaldo, Larry Wasserman. Investigating the Cosmic Web with Topological Data Analysis. \textit{American Astronomical Society Meeting.} (AAS) 2018.

%----------------------------------------------------------------------------------------
%	PROFESSIONAL EXPERIENCE SECTION
%----------------------------------------------------------------------------------------

\section{Industry \\ Experience}

\textbf{Facebook Applied Machine Learning} (AML) \hfill 2016-2017 \\
Visiting engineer building tools in computer vision.

\textbf{Lattice Data} \hfill 2016 \\
Software engineer building weakly supervised classifiers.

\textbf{Invrea} (Inverse Reasoning) \hfill 2015-2017 \\
\url{http://invrea.com} \\
Co-founder a venture for a probabilistic programming language in Excel.

\textbf{YHack} \hfill 2013-2016 \\
\url{https://www.yhack.org} \\
Co-founded an international collegiate hackathon.

\textbf{Ionis Pharmaceuticals} \hfill 2013 \\
Data science Intern building classifiers for drug design.

%----------------------------------------------------------------------------------------

\end{resume}
\end{document}